\documentclass[13pt,largemargins]{homework}

\usepackage{lipsum}
\usepackage{amssymb}
\usepackage[utf8]{inputenc}
\usepackage{lmodern}
\usepackage{amsfonts}
\usepackage{hyperref}
\usepackage{bbm}
\usepackage{amsmath}
\usepackage{epstopdf}
\usepackage{enumitem}
\usepackage{subcaption}
\usepackage{bbm}

\setcounter{MaxMatrixCols}{15}


\begin{document}
\section %ESERCIZIO 3
\begin{enumerate}[label=(\alph*)]
%inserire grafo ???
\item %3a
Il grafo in questione è indiretto e aciclico. Il vettore dei vertici, ordinati in ordine alfabetico, risulta essere:
\begin{gather*}
v = \{ Acciaiuoli, Albizzi, Barbadori, Bischeri, Castellani,\\
Ginori, Guadagni, Lamberteschi, Medici, Pazzi,\\ Peruzzi, Ridolfi, Salviati, Strozzi, Tornabuoni \}\\
\end{gather*}

Si procede pertanto scrivendo la matrice di adiacenza \(W\) ed il vettore dei gradi \(w_i\). 

\[W=\begin{bmatrix}
	0 & 0 & 0 & 0 & 0 & 0 & 0 & 0 & 1 & 0 & 0 & 0 & 0 & 0 & 0\\
	0 & 0 & 0 & 0 & 0 & 1 & 1 & 0 & 1 & 0 & 0 & 0 & 0 & 0 & 0\\
	0 & 0 & 0 & 0 & 1 & 0 & 0 & 0 & 1 & 0 & 0 & 0 & 0 & 0 & 0\\
	0 & 0 & 0 & 0 & 0 & 0 & 1 & 0 & 0 & 0 & 1 & 0 & 0 & 1 & 0\\
	0 & 0 & 1 & 0 & 0 & 0 & 0 & 0 & 0 & 0 & 1 & 0 & 0 & 1 & 0\\
	0 & 1 & 0 & 0 & 0 & 0 & 0 & 0 & 0 & 0 & 0 & 0 & 0 & 0 & 0\\
	0 & 1 & 0 & 1 & 0 & 0 & 0 & 1 & 0 & 0 & 0 & 0 & 0 & 0 & 1\\
	0 & 0 & 0 & 0 & 0 & 0 & 1 & 0 & 0 & 0 & 0 & 0 & 0 & 0 & 0\\
	1 & 1 & 1 & 0 & 0 & 0 & 0 & 0 & 0 & 0 & 0 & 1 & 1 & 0 & 1\\
	0 & 0 & 0 & 0 & 0 & 0 & 0 & 0 & 0 & 0 & 0 & 0 & 1 & 0 & 0\\
	0 & 0 & 0 & 1 & 1 & 0 & 0 & 0 & 0 & 0 & 0 & 0 & 0 & 1 & 0\\
	0 & 0 & 0 & 0 & 0 & 0 & 0 & 0 & 1 & 0 & 0 & 0 & 0 & 1 & 0\\
	0 & 0 & 0 & 0 & 0 & 0 & 0 & 0 & 1 & 1 & 0 & 0 & 0 & 0 & 0\\
	0 & 0 & 0 & 1 & 1 & 0 & 0 & 0 & 0 & 0 & 1 & 1 & 0 & 0 & 0\\
	0 & 0 & 0 & 0 & 0 & 0 & 1 & 0 & 1 & 0 & 0 & 0 & 0 & 0 & 0
	\end{bmatrix}\]
	
	\begin{gather*}
	\omega_i = (1, 3, 2, 3, 3, 1, 4, 1, 6, 1, 3, 2, 2, 4, 2)\\
	\end{gather*}
La misura di centralità invariante è unica e vale: 
\begin{gather*}
	\pi_i = ( \frac{1}{38}, \frac{3}{38}, \frac{1}{19}, \frac{3}{38}, \frac{3}{38}, \frac{1}{38}, \frac{2}{38}, \frac{1}{38}, \frac{3}{19}, \frac{1}{38}, \frac{3}{38}, \frac{1}{19}, \frac{1}{19}, \frac{2}{19}, \frac{1}{19} ).\\
\end{gather*}
	
	Essendo il grafo fortemente connesso e aperiodico si può concludere che la dinamica di averaging di French-De Groot converge al valore di consenso come nell'esercizio precedente: 
\[\lim_{t \to \infty}x_i(t)=\pi ' x(0)= \sum_i 	\pi_i x_i(0), \forall x(0)\] e pertanto, essendo $x(0)=[0,0,0,0,0,0,0,0,1,0,0,0,0,-1,0]'$ il vettore delle condizioni iniziali, si ottiene che il valore finale di consenso vale: 
\[\lim_{t \to \infty}x_i(t)= (1)\frac{3}{19}+(-1)\frac{2}{19}=\frac{1}{19}\]

\item %3b 
%qui ci va il codice Matlab del secondo punto 
	
\item %3c
L'insieme dei nodi stubborn $S=\{Castellani, Guadagni, Medici, Strozzi\}$ è globalmente raggiungibile. Sia $D=diag(\omega_i)$ e \(P\) la matrice dei pesi normalizzata
\[P=D^{-1}W=\begin{bmatrix} 
0 & 0 & 0 & 0 & 0 & 0 & 0 & 0 & 1 & 0 & 0 & 0 & 0 & 0 & 0\\
	0 & 0 & 0 & 0 & 0 & \frac{1}{3} & \frac{1}{3} & 0 & \frac{1}{3} & 0 & 0 & 0 & 0 & 0 & 0\\
	0 & 0 & 0 & 0 & \frac{1}{2} & 0 & 0 & 0 & \frac{1}{2} & 0 & 0 & 0 & 0 & 0 & 0\\
	0 & 0 & 0 & 0 & 0 & 0 & \frac{1}{3} & 0 & 0 & 0 & \frac{1}{3} & 0 & 0 & \frac{1}{3} & 0\\
	0 & 0 & \frac{1}{3} & 0 & 0 & 0 & 0 & 0 & 0 & 0 & \frac{1}{3} & 0 & 0 & \frac{1}{3} & 0\\
	0 & 1 & 0 & 0 & 0 & 0 & 0 & 0 & 0 & 0 & 0 & 0 & 0 & 0 & 0\\
	0 & \frac{1}{4} & 0 & \frac{1}{4} & 0 & 0 & 0 & \frac{1}{4} & 0 & 0 & 0 & 0 & 0 & 0 & \frac{1}{4}\\
	0 & 0 & 0 & 0 & 0 & 0 & 1 & 0 & 0 & 0 & 0 & 0 & 0 & 0 & 0\\
	\frac{1}{6} & \frac{1}{6} & \frac{1}{6} & 0 & 0 & 0 & 0 & 0 & 0 & 0 & 0 & \frac{1}{6} & \frac{1}{6} & 0 & \frac{1}{6}\\
	0 & 0 & 0 & 0 & 0 & 0 & 0 & 0 & 0 & 0 & 0 & 0 & 1 & 0 & 0\\
	0 & 0 & 0 & \frac{1}{3} & \frac{1}{3} & 0 & 0 & 0 & 0 & 0 & 0 & 0 & 0 & \frac{1}{3} & 0\\
	0 & 0 & 0 & 0 & 0 & 0 & 0 & 0 & \frac{1}{2} & 0 & 0 & 0 & 0 & \frac{1}{2} & 0\\
	0 & 0 & 0 & 0 & 0 & 0 & 0 & 0 & \frac{1}{2} & \frac{1}{2} & 0 & 0 & 0 & 0 & 0\\
	0 & 0 & 0 & \frac{1}{4} & \frac{1}{4} & 0 & 0 & 0 & 0 & 0 & \frac{1}{4} & \frac{1}{4} & 0 & 0 & 0\\
	0 & 0 & 0 & 0 & 0 & 0 & \frac{1}{2} & 0 & \frac{1}{2} & 0 & 0 & 0 & 0 & 0 & 0
\end{bmatrix}\]
Pertanto, il vettore di equilibrio della dinamica di averaging è dato dalla risoluzione del seguente sistema lineare
\begin{equation}
	\begin{cases}
		x_5 = x_7 = x_{14} = -1\\
		x_9 = +1\\
		x_i = \sum\limits_{j} P_{ij}x_j, \j=\{1,2,3,4,6,8,10,11,12,13,15\}
	\end{cases}
\end{equation}	

\begin{equation}
	\begin{cases}
		x_5 = x_7 = x_{14} = -1\\
		x_9 = +1\\
		x_1 = x_9\\		
		x_2 = \frac{1}{3}x_6 + \frac{1}{3}x_7 + \frac{1}{3}x_9\\
		x_3 = \frac{1}{2}x_5 + \frac{1}{3}x_9\\
		x_4 = \frac{1}{3}x_7 + \frac{1}{3}x_{11} + \frac{1}{3}x_{14}\\
		x_6 = x_2\\
		x_8 = x_7\\
		x_{10} = x_{13}\\
		x_{11} = \frac{1}{3}x_4 + \frac{1}{3}x_5 + \frac{1}{3}x_{14}\\		
		x_{12} = \frac{1}{2}x_9 + \frac{1}{3}x_{14}\\
		x_{13} = \frac{1}{2}x_9 + \frac{1}{2}x_{10}\\
		x_{15} = \frac{1}{2}x_7 + \frac{1}{2}x_9\\	
	\end{cases}
\end{equation}	

La soluzione del sistema è
\begin{equation}
x(0)=[1,0,0,-1,-1,0,-1,-1,1,1,-1,0,1,-1,0]'
\end{equation}
da cui si deduce che Acciaiuoli, Pazzi e Salviati avranno la stessa opinionde dei Medici pari a $+1$ mentre Bischeri, Castellani, Guadagni, Lamberteschi, Peruzzi e Strozzi condivideranno l'opinione $-1$ ed i restanti Albizzi, Barbadori, Ginori, Ridolfi e Tornabuoni manterranno l'opinione iniziale pari a $0$.  

\item %3d 
%Page Rank centrality


\end{enumerate}

\end{document}